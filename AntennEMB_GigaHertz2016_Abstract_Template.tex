

\documentclass[10pt,final,conference,a4paper,twocolumn]{IEEEtran_AntennEMB_GigaHertz2016}
\usepackage{graphicx}



\usepackage[swedish,english]{babel}
\usepackage[bottom]{footmisc}



% The document begins here
\begin{document}

% paper title
\title{Continuous complex permittivity extraction from water-glucose solutions using a resonant microwave cavity at 300 MHz}

% author names and affiliations
\author{Matthias Carlsson$^1$, Gustav Eriksson$^1$, Jens Ljungberg$^1$ and Dragos Dancila$^1$ \\
\em \small \textbf{1:} Division of Solid-State Electronics, Uppsala University, SE-751 21 Uppsala, Sweden\\
\small matthias.carlsson.5845@student.uu.se, gustav.eriksson.0648@student.uu.se, \\
\small jens.ljungberg.2853@student.uu.se and dragos.dancila@angstrom.uu.se
}

\maketitle
\section{INTRODUCTION}
One particular field which has been recently studied is to monitor the permittivity change of blood at different glucose levels \cite{c2}. 
 
 In this paper a method based on the small cavity perturbation theory \cite{c3} is proposed to measure complex permittivity noninvasively and continuously. In the proposed setup the liquid is circulating through a strong electric field (along the z-axis in Fig \ref{fig:sim}). The frequency response around the cavity's resonance frequency is measured by a Vector Network Analyzer (VNA). The method of determining the permittivity from the frequency response can be found in previous papers \cite{c2}\cite{c1} but with different setups, higher frequencies and higher water-glucose concentrations compared to what are presented in these papers.
 
 The setup is powerful enough to discern a trend in glucose concentrations ranging from 50-200mg/dl, the typical glucose range of human blood. The method is generic and can be usedin a variety of situations to monitor continuously and non-invasively small permittivity changes for the wear of lube oils, alcohol etc.

\begin{figure}[b]
 	\centering
 	\includegraphics[width=1.0\columnwidth]{EHfield.png}
 	\caption{A simulation of the cavity with HFFS}
 	\label{fig:sim}
 \end{figure}
\section{RESULTS}
% \begin{figure}[b]
%	\centering 
%	\includegraphics[width=1.0\columnwidth]{pumpgluzoomERE.png}
%	\caption{Re. part of c. permittivity, $\epsilon'$ vs concentration. Shifts in $\epsilon''$ due to small temperature variations. The circles are the original data (No temperature compensation) and the stars are the temperature shifted points}
%	\label{fig:ere}
%\end{figure}
To evaluate the setup, measurements were made of water-glucose solutions with different glucose concentrations. 168(05) $\mu$l of glucose solution with a concentration of 1g/10ml was mixed with 5.00(50) dl DI-water. Between each measurement the concentration was increased by 10 mg/dl. The results are shown in Fig. \ref{fig:eim} where the original measurements are shown as circles and a calculated temperature shift corresponding to a temperature difference between 0.000(03) to $\pm$0.007(03)$^o$C are shown as stars.

\begin{figure}[t]
	\centering
	\includegraphics[width=1.0\columnwidth]{pumpgluzoomEIM.png}
	\caption{Real and Imaginary part of c. permittivity, $\epsilon'$ and $\epsilon''$ respectively vs concentration. Shifts in $\epsilon'$ and $\epsilon''$ due to small temperature variations. The circles are the original data (No temperature compensation) and the stars are the temperature shifted points}
	\label{fig:eim}
\end{figure}






%\section{CITING PREVIOUS WORK} 
%When referencing a journal article~\cite{hansen84}, a conference digest
%article~\cite{richter01} or a book~\cite{stutzman98} use the usual \LaTeX\ biblio\-graphy environments.



\section{CONCLUSIONS}
In this paper we implement and evaluate a method of measuring the permittivity continuously and noninvasively the complex permittivity of a fluid using a cavity resonator. We show that the method is accurate enough to result in a trend in the medical range.

For improvements, the uncertainty in the permittivity can be reduced by using better reference values and improved concentration and temperature control.
%\addtolength{\textheight}{-18cm}   % This command serves to balance the column lengths
% on the last page of the document manually. It shortens
% the textheight of the last page by a suitable amount.
% This command does not take effect until the next page
% so it should come on the page before the last. Make
% sure that you do not shorten the textheight too much.




%\section*{ACKNOWLEDGEMENT}   
%The authors wish to acknowledge the assistance and support of all those who
%contribute to...

% references section

% NOTE: BibTeX documentation can be easily obtained at:
% http://www.ctan.org/tex-archive/biblio/bibtex/contrib/doc/

% You can use a bibliography generated by BibTeX as a .bbl file:
%\bibliographystyle{IEEEtran}
%\bibliography{IEEEabrv,mybib}

% <OR>

% manually copy the resulting .bbl file into this document
% and set the second argument of \begin to the number of references
% (used to reserve space for the reference number labels box)
\begin{thebibliography}{99}
	
	
	
	\bibitem{c2} Hofmann, Maximilian and Fischer, Georg and Weigel, Robert and Kissinger, Dietmar, "Microwave-Based Noninvasive Concentration Measurements for Biomedical Applications", in 2013, IEEE T. on M. T. and T., Volume 61, number 5, pages 2195-2204
	
	\bibitem{c3} D. M. Pozar, Microwave Engineering, Chapter 6: Microwave Resonators. John Wiley \& Sons,
	Inc, 2012.
	
	\bibitem{c1} V. Turgul and I. Kale, "Characterization of the complex permittivity of glucose/water solutions for noninvasive RF/Microwave blood glucose sensing," in 2016, . DOI: 10.1109/I2MTC.2016.7520546.
\end{thebibliography}

% A crude (but effective) way to achieve equally high columns (which looks nicer) on the last page is to shorten the page length of the first % column of the
% last page via \enlargethispage{-X.Xcm} where X.X is the value you find that makes the columns equal.
% This hack will have to be adjusted if the document is altered meaning you will have to tweak the value of X.X iteratively and might
% also change the actual positioning of this command within your text.
% The only way to get around this problem is writing a routine inside the class file which actually calculates the print space on the last
% page and adjusts iteratively (but automatically!) the text height on this page.

%\enlargethispage{-18.5cm} %this shortens the last page so that both columns have the same height (use this after you completely finished your paper)
\end{document}
